\documentclass[12pt]{article}

% Codificação e Idioma
\usepackage[utf8]{inputenc}
\usepackage[portuguese]{babel}

% Formatação de Página e Estilo de Texto
\usepackage[a4paper, left=3cm, right=2cm, top=3cm, bottom=2cm]{geometry}
\usepackage{parskip}
\usepackage{setspace}
\onehalfspacing
\usepackage{csquotes}

% Links
\usepackage{xcolor}
\usepackage[hidelinks]{hyperref}

% Matemática
\usepackage{amsmath, amssymb, amsfonts}
\usepackage{cancel}
\usepackage{tikz}

% Bibliografia (ABNT)
\usepackage[backend=biber, style=abnt, language=portuguese, natbib=true]{biblatex}
\addbibresource{ref.bib}


% Outros Pacotes
\usepackage{graphicx}
\usepackage{ascii}
\usepackage{listings}
\usepackage{enumitem}
\usepackage{url}

% Definir título da bibliografia
\renewcommand{\refname}{Referências}  

\title{Relatório do Trabalho de conclusão de curso \\ \large{MAP2429 - Trabalho de Formatura em Matemática Aplicada}}
\date{Setembro de 2025}

\author{
\textbf{Aluno:} Lucas Amaral Taylor (IME-USP)\\
\textbf{Orientador:} Prof. Dr. Breno Raphaldini Ferreira da Silva (IME-USP)
}

\begin{document}
\maketitle

O presente relatório tem como objetivo apresentar as atividades desenvolvidas entre os meses de junho a setembro de 2025, no contexto do Trabalho de Conclusão de Curso.

\begin{enumerate}
	\item \textbf{Simulação de modelo de SDE e introdução à linguagem Julia.} Entre os meses de junho e julho, foi realizado um estudo baseado no modelo apresentado na página 169 de \citet{Pavliotis2008}. O modelo consistia em um movimento browniano acoplado à variável $y$ do sistema de Lorenz 63. Essa simulação teve dois objetivos principais: trabalhar com um modelo simplificado que permitisse a transição do regime determinístico para o estocástico (algo que pretendemos realizar futuramente com o modelo de Lorenz 80), e introduzir a biblioteca \textit{SciML: Differentiable Modeling and Simulation Combined with Machine Learning} \citep{Rackauckas2017}. Todo o material desenvolvido nesta atividade está disponível no \textcolor{blue}{\href{https://github.com/lucasamtaylor01/Lorenz80_SDE/tree/portuguese-version/05_WIENER_ACOPLADO}{Github}}.
	
	\item \textbf{Introdução às equações diferenciais estocásticas.} Entre agosto e setembro, foram feitas leituras introdutórias para familiarizar o estudante com o tema de equações diferenciais estocásticas. As principais referências utilizadas foram os livros de \citet{Pavliotis2014} e \cite{Evans2014}, ambos com abordagem introdutória. Os temas abordados incluíram: introdução aos processos estocásticos, fundamentos de estatística, processos difusivos, equações diferenciais estocásticas, equações de Fokker-Planck e o processo de Wiener. Durante esse período, orientador e aluno realizaram reuniões semanais para discutir dúvidas teóricas e aprofundar os conceitos estudados.

	\item \textbf{Escrita do Trabalho de Conclusão de Curso.} Em agosto, iniciou-se a redação da monografia do TCC. A atividade segue em andamento e está sendo desenvolvida no repositório disponível no \textcolor{blue}{\href{https://github.com/lucasamtaylor01/Lorenz80_SDE/tree/portuguese-version/00_TCC}{Github}}. O primeiro capítulo apresenta uma introdução ao modelo determinístico de Lorenz 80, suas características teóricas e simulações computacionais. Estas últimas estão disponível no \textcolor{blue}{\href{https://github.com/lucasamtaylor01/Lorenz80_SDE/tree/portuguese-version/06_LORENZ_80}{Github}}.
\end{enumerate}

\newpage
\nocite{*}
\printbibliography
\end{document}
