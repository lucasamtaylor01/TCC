\documentclass[12pt]{article}
% Encoding and Language
\usepackage[utf8]{inputenc}
\usepackage[portuguese]{babel}
% Page Formatting and Text Style
\usepackage[a4paper, left=3cm, right=2cm, top=3cm, bottom=2cm]{geometry}
\usepackage{parskip}
\usepackage{setspace}
\onehalfspacing
\usepackage{csquotes}
% Links
\usepackage{xcolor}
\usepackage[hidelinks]{hyperref}
% Mathematics
\usepackage{amsmath, amssymb, amsfonts}
\usepackage{cancel}
\usepackage{tikz}
% Bibliography (ABNT)
\usepackage[backend=biber, style=abnt, language=portuguese, natbib=true]{biblatex}
\addbibresource{ref.bib}

% Other Packages
\usepackage{graphicx}
\usepackage{ascii}
\usepackage{listings}
\usepackage{enumitem}
\usepackage{url}
% Set bibliography title
\renewcommand{\refname}{References}
\title{Final Course Project Report \\ \large{MAP2419 - Introduction to Graduation Project}}
\date{April 2025}
\author{
\textbf{Student:} Lucas Amaral Taylor (IME-USP)\\
\textbf{Advisor:} Prof. Dr. Breno Raphaldini Ferreira da Silva (IME-USP)
}
\begin{document}
\maketitle
The purpose of this report is to present the activities carried out during the month of April 2025, in the context of the Final Course Project.
\begin{enumerate}
    \item \textbf{Definition of the topic.} The provisional topic defined for the project was “A Stochastic Approach to the L80 Model”\footnote{Title subject to change}. The project is based on the study of the article by \citet{Chekroun2021};
          
    \item \textbf{Bibliographic survey.} The main theoretical references that will underpin the development of the work were identified and selected. All are available on the \textbf{Reference} page of the report;
          
    \item \textbf{Creation of the repository on \textit{GitHub}.} In order to organize the tasks and centralize the project materials, a repository was created on \textit{GitHub}, available at:
                 \textcolor{blue}{\href{https://github.com/lucasamtaylor01/Lorenz80_SDE}{https://github.com/lucasamtaylor01/Lorenz80_SDE}};
              
\item \textbf{Reading of the base articles.} An initial reading of two articles fundamental to the theoretical basis of the project was carried out: \citet{Chekroun2017} and \citet{Chekroun2021};
          
    \item \textbf{Seminar presentation on Mori-Zwanzig Formalism.} As part of the theoretical deepening activities, an introductory seminar on Mori-Zwanzig Formalism was prepared and presented, based on Chapter 09 of the book \textit{Stochastic Tools in Mathematics and Science} \citep{Chorin2013}. The materials used are available in the repository:
                  \textcolor{blue}{\href{https://github.com/lucasamtaylor01/Lorenz80_SDE/tree/master/03_SEMINARIO_MZ}{https://github.com/lucasamtaylor01/Lorenz80_SDE/tree/master/03_SEMINARIO_MZ}}.
\end{enumerate}

\newpage
\nocite{*}
\printbibliography
\end{document}
