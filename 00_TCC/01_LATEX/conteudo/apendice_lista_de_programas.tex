\chapter{Programas}  \label{app:apendice-lista-de-programas}

\definecolor{codebg}{RGB}{247,248,250}
\definecolor{codenumber}{HTML}{7F8C8D}
\definecolor{codekw}{HTML}{005CC5}
\definecolor{codestr}{HTML}{032F62}
\definecolor{codecm}{HTML}{6A737D}
\definecolor{codetype}{HTML}{D73A49}
\definecolor{codemacro}{HTML}{8A2BE2}

% Julia para listings
\lstdefinelanguage{Julia}{
  alsoletter={@},
  morekeywords={
    abstract,break,catch,const,continue,do,else,elseif,end,export,false,finally,
    for,function,global,if,import,let,local,macro,module,mutable,primitive,
    quote,return,struct,true,try,using,where,while,begin
  },
  morekeywords=[2]{Int,Int8,Int16,Int32,Int64,UInt,UInt8,UInt16,UInt32,UInt64,
    Float16,Float32,Float64,Complex,ComplexF64,Bool,Char,String,Nothing,Any,
    Vector,Matrix,Dict,Tuple,Union,SubArray,AbstractArray,Range},
  morekeywords=[3]{@time,@views,@inbounds,@simd,@threads,@btime,@benchmark,
    @show,@assert,@inline,@code_warntype,@test},
  sensitive=true,
  morecomment=[l]\#,
  morestring=[b]",
}

% Estilo base
\lstdefinestyle{jlstyle}{
  language=Julia,
  backgroundcolor=\color{codebg},
  basicstyle=\ttfamily\small,
  numbers=left,
  numberstyle=\scriptsize\color{codenumber},
  numbersep=8pt,
  showstringspaces=false,
  tabsize=2,
  columns=fullflexible,
  keepspaces=true,
  breaklines=true,
  frame=single,
  rulecolor=\color{codebg},
  keywordstyle=\bfseries\color{codekw},
  keywordstyle=[2]\color{codetype},
  keywordstyle=[3]\color{codemacro},
  commentstyle=\itshape\color{codecm},
  stringstyle=\color{codestr},
  upquote=true,
  % Alguns símbolos/Unicode comuns em Julia
  literate=
    {→}{{$\to$}}1 {←}{{$\leftarrow$}}1
    {≤}{{$\le$}}1 {≥}{{$\ge$}}1
    {α}{{$\alpha$}}1 {β}{{$\beta$}}1 {γ}{{$\gamma$}}1
    {π}{{$\pi$}}1 {√}{{$\sqrt{\phantom{x}}$}}1
    {λ}{{$\lambda$}}1 {ϵ}{{$\varepsilon$}}1
    {σ}{{$\sigma$}}1 {²}{{$^2$}}1
    {Δ}{{$\Delta$}}1
}

% Ambiente e macro inline
\lstnewenvironment{juliacode}[1][]%
  {\lstset{style=jlstyle,#1}}{}

\newcommand{\jinline}[1]{\lstinline[style=jlstyle]!#1!}

\section{Código do modelo Lorenz 80 determinístico}
\begin{juliacode}[caption={Simulação do modelo Lorenz 80}]
using DifferentialEquations, ModelingToolkit, CSV, DataFrames

@parameters t
@variables x[1:3](t) y[1:3](t) z[1:3](t)
D = Differential(t)

a = [1.0, 1.0, 3.0]
b = [0.5 * (a[1] - a[2] - a[3]), 0.5 * (a[2] - a[3] - a[1]), 0.5 * (a[3] - a[1] - a[2])]
c = sqrt(3/4)
h = [-1.0, 0.0, 0.0]
f = [0.1, 0.0, 0.0]
g_0, kappa_0, nu_0 = 8.0, 1/48, 1/48

eqs = [
    D(x[i]) ~ (
        a[i]*b[i]*x[(i % 3) + 1]*x[((i+1) % 3) + 1]
        - c*(a[i]-a[((i+1) % 3) + 1])*x[(i % 3) + 1]*y[((i+1) % 3) + 1]
        - 2*c^2*y[i]*y[((i+1) % 3) + 1]
        - nu_0*a[i]^2*x[i]
        + a[i]*y[i] - a[i]*z[i]
    ) / a[i] for i in 1:3
]
append!(eqs, [
    D(y[i]) ~ (
        -a[((i+1) % 3) + 1]*b[((i+1) % 3) + 1]*x[(i % 3) + 1]*y[((i+1) % 3) + 1]
        - a[(i % 3) + 1]*b[(i % 3) + 1]*y[(i % 3) + 1]*x[((i+1) % 3) + 1]
        - a[i]*x[i] - nu_0*a[i]^2*y[i]
    ) / a[i] for i in 1:3
])
append!(eqs, [
    D(z[i]) ~ (
        -b[((i+1) % 3) + 1]*x[(i % 3) + 1]*(z[((i+1) % 3) + 1]-h[((i+1) % 3) + 1])
        - b[(i % 3) + 1]*(z[(i % 3) + 1]-h[(i % 3) + 1])*x[((i+1) % 3) + 1]
        + c*y[(i % 3) + 1]*(z[((i+1) % 3) + 1]-h[((i+1) % 3) + 1])
        - c*(z[(i % 3) + 1]-h[(i % 3) + 1])*y[((i+1) % 3) + 1]
        + g_0*a[i]*x[i] - kappa_0*a[i]*z[i] + f[i]
    ) for i in 1:3
])

@mtkbuild sys = ODESystem(eqs, t)

x0, y0, z0 = [0.1, 0.0, 0.0], [0.1, 0.0, 0.0], [0.1, 0.0, 0.0]
u0 = vcat(x0, y0, z0)
tspan = (0.0, 8*10.0) # 8*numero_de_dias

sol = solve(ODEProblem(sys, u0, tspan), Tsit5(); abstol=1e-6, reltol=1e-8, saveat=0.01)

U = Array(sol)
df = DataFrame(time = sol.t, x1 = U[1,:], x2 = U[2,:], x3 = U[3,:], y1 = U[4,:], y2 = U[5,:], y3 = U[6,:],z1 = U[7,:], z2 = U[8,:], z3 = U[9,:])

cd(@__DIR__)  
isdir("data") || mkdir("data")
CSV.write("data/solucao.csv", df)
\end{juliacode}

\section{Exemplo do capítulo 02}

\begin{juliacode}[caption={Modelo determinístico browniano acoplado}]
using DifferentialEquations, ModelingToolkit, Plots, BenchmarkTools, CSV, DataFrames

@independent_variables t
@variables x(t)=0.1 y1(t)=0.01 y2(t)=0.01 y3(t)=0.01
@parameters ϵ=0.01 λ=1.0
D = Differential(t)

eqs = [
    D(x) ~  x - x^3 + (λ/ϵ) * y2,
    D(y1) ~ (10 / ϵ^2) * (y2 - y1),
    D(y2) ~ (1 / ϵ^2) * (28*y1 - y2 - y1*y3),  
    D(y3) ~ (1 / ϵ^2) * (y1*y2 - (8/3)*y3)
]

@mtkbuild sys = ODESystem(eqs, t)

tspan = (0.0, 1.0)
prob = ODEProblem(sys, [], tspan)
solucao_deterministico = solve(prob, Tsit5(), reltol=1e-6, abstol=1e-7, saveat=1e-3)

x_vals = solucao_deterministico[x]
y2_vals = solucao_deterministico[y2]

df = DataFrame(t = solucao_deterministico.t, y2 = y2_vals, x = x_vals)
cd(@__DIR__)  
CSV.write("data/deterministico.csv", df)
\end{juliacode}

\begin{juliacode}[caption={Cálculo do sigma}]
using DataFrames, Plots, CSV, Statistics, StatsBase, DSP

cd(@__DIR__)
df = CSV.read("data/deterministico.csv", DataFrame)

y2_vals = df[:, 2]
t_vals = df[:,1]
Δt = 1e-3

y2_centered = y2_vals .- mean(y2_vals)

acor = xcorr(y2_centered; scaling = :coeff)
n = length(y2_centered)
acor_pos = acor[n:end]  # lags ≥ 0

λ = 1.0

σ² = 2 * λ^2 * sum(acor_pos) * Δt
σ = sqrt(σ²)

println(σ)
\end{juliacode}

\begin{juliacode}[caption={Modelo estocastico browniano acoplado}]
using DifferentialEquations, Plots

sigma = 8.880502080440465
x0 = 0.1
tspan = (0.0, 1.0)

f1(X, p, t) = X - X^3
f2(X, p, t) = p

W = WienerProcess(0.0, 0.0)
prob = SDEProblem(f1, f2, x0, tspan, sigma; noise = W)
solucao_estocastico = solve(prob, EM(), dt = 1e-3)

df = DataFrame(t = solucao_estocastico.t, x = solucao_estocastico.u)
cd(@__DIR__)  
CSV.write("data/estocastico.csv", df)
\end{juliacode}