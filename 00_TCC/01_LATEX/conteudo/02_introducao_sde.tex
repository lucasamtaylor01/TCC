\chapter{Introdução às equações diferenciais estocásticas} \label{cap:ch02_introducao_a_sde}

\section{Introdução} \label{sec:ch02_introducao}

\section{Noções gerais de estatística}
Nesta seção, vamos apresentar as principais definições e teoremas relacionados a probabilidade e estatística que constroem a base necessária para compreensão de equações diferenciais estocásticas. Todos os conceitos apresentados estão detalhados em \citet{Pavliotis2014} e \citet{Evans2014}.

\begin{itemize}
    \item \textbf{Variável aleatória.} Seja $(\Omega, \mathcal{F}, \mathbb{P})$ um espaço de probabilidade e $(E, \mathcal{G})$ um espaço mensurável. Definimos uma variável aleatória como uma função mensurável da forma
    \begin{equation*}
        X : (\Omega, \mathcal{F}) \to (E, \mathcal{G}).
    \end{equation*}
    \item \textbf{Processo Estocástico.} Definimos um processo estocástico como sendo uma coleção de variáveis aleatórias $\{X(t) : t \in T\}$, onde $T$ é um conjunto ordenado.
    \item \textbf{Esperança condicional.}
    \item \textbf{Processos de Markov.}
    \item \textbf{Equações de Champman-Kolmogorov.}
    \item \textbf{Processo de difusão.}
    \item \textbf{\textit{Forward and Backward Kolmogorov Equation}}
\end{itemize}

\section{Movimento Browniano (Processo de Wiener)} \label{sec:ch02-movimento-browniano}

\subsection{Desenvolvimento histórico} \label{subsec:ch02-motivacao-movimento-browniano}
O movimento browniano tem como origem o problema físico: estudar o movimento de grãs de polém. Em 1827, Robert Brown


% Em 1827, ao olhar através de um microscópio partículas encontradas em grãos de pólen na água,o biólogo Robert Brown observou que as partículas se moviam através da água, mas não foi capaz de determinar os mecanismos que causaram este movimento. Assim, foi o primeiro a observar cientificamente o movimento que achou se tratar de uma nova forma de vida, pois ainda não se tinha completa ciência da existência de moléculas, e as partículas pareciam descrever movimentos por vontade própria.

% Jan Ingenhousz também fez algumas observações do movimento irregular de poeira de carbono em álcool em 1765. Porém, a primeira pessoa a descrever a matemática por trás do movimento Browniano foi Thorvald N. Thiele em 1880 em um artigo no método dos menores quadrados. Isto foi seguido independentemente por Louis Bachelier em 1900 em sua tese de PhD "A Teoria da Especulação".

% Átomos e moléculas , posteriormente foram teorizados como os constituintes da matéria e, muitas décadas depois, Albert Einstein publicou um artigo em 1905 que explicava em detalhes precisos como o movimento que Brown tinha observado era o resultado do pólen sendo movido por moléculas de água individuais. Esta explicação deste fenômeno de transporte serviu como a confirmação definitiva de que átomos e moléculas realmente existem, e foi ainda verificada experimentalmente por Jean Baptiste Perrin, em 1908. Perrin foi agraciado com o Prêmio Nobel de Física em 1926 "por seu trabalho sobre a estrutura descontínua da matéria" (Einstein tinha recebido o prêmio cinco anos antes "por seus serviços à física teórica", com citação específica de uma pesquisa diferente). Sendo então que a direção da força de bombardeamento atômico está constantemente mudando, e em diferentes momentos da partícula é atingido mais de um lado do que o outro, levando à natureza aparentemente aleatória do movimento. 

\subsection{Definição e Propriedades} \label{subsec:ch02-propriedades-movimento-browniano}
Um processo estocástico real, denotado por $W(\cdot)$ é chamado de processo de Wiener quando satisfaz as seguintes propriedades:
\begin{enumerate}
    \item $W(0) = 0$, quase certamente;
    \item Para todo $t \geq s \geq 0$, tem-se que $W(t) - W(s) \sim \text{Normal}(0,t-s)$;
    \item $W(\cdot)$ possui incrementos independentes, isto é, para $0 < t_1 < t_2 < \cdots < t_n$, as variáveis aleatórias
    \begin{equation*}
        W(t_1), \; W(t_2) - W(t_1), \; \ldots, \; W(t_n) - W(t_{n-1}) \text{  são independentes.}
    \end{equation*}
\end{enumerate}

\section{Equações diferenciais estocásticas}




\section{Exemplo}

A seguir, apresentemos um exemplo retirado de \citet{Pavliotis2008} de uma aproximação de um sistema determinístico para um sistema estocástico. Tal exemplo é relevante, pois trata-se de uma abordagem simplificada do que vamos realizar com o modelo de Lorenz 80.

O exemplo trata-se de um movimento browniano, assim como apresentado em na seção \ref{subsec:ch02-propriedades-movimento-browniano} acoplado ao sistema Lorenz 63 apresentado em \ref{eq:ch01-lorenz63} a partir das variáveis $y=(y_1, y_2, y_3)^\intercal$. O exemplo em questão é expresso por:

\begin{equation}\label{eq:ch02-exemplo-de-aprox}
    \begin{aligned}
        \frac{dx}{dt} &= x - x^{3} + \frac{\lambda}{\varepsilon} y_{2}, \\
        \frac{dy_{1}}{dt} &= \frac{10}{\varepsilon^{2}} (y_{2} - y_{1}), \\
        \frac{dy_{2}}{dt} &= \frac{1}{\varepsilon^{2}} \left( 28 y_{1} - y_{2} - y_{1} y_{3} \right), \\
        \frac{dy_{3}}{dt} &= \frac{1}{\varepsilon^{2}} \left( y_{1} y_{2} - \tfrac{8}{3} y_{3} \right).
    \end{aligned}
\end{equation}

\textbf{EXEMPLICAR POR QUE PODEMOS APROXIMAR}

Podemos aproximar o modelo para sua versão estocástica forma de Itô:
\begin{equation}
    \frac{dX}{dt} = X - X^{3} + \sigma \frac{dW}{dt},
\end{equation}

Onde $\sigma$ é expresso por:
\begin{equation}
    \sigma^{2} = 2 \lambda^{2} \int_{0}^{\infty} \frac{1}{T}  \left( \lim_{T \to \infty} \int_{0}^{T} \psi^{s}(y)\, \psi^{t+s}(y)\, ds \right) dt.
\end{equation}

A partir das equações apresentadas, podemos realizar simulações computacionais. Novamente, as simulações foram realizadas com o uso da biblioteca \textit{SciML} \citep{Rackauckas2017} e os programas que geraram os dados estão no apêndice \ref{app:apendice-lista-de-programas}. 

\textbf{VER O NEGÓCIO DA SEED}

\begin{figure}[H]
    \centering
    \subfloat[Browniano acoplado determinístico.]{
        \includegraphics[width=0.45\textwidth]{00_TCC/01_LATEX/figuras/ch02_introducao_a_sde/deterministico_plot.png}
        \label{fig:ch02-exemplo-histograma-deterministico}
    }
    \hfill
    \subfloat[Browniano acoplado estocástico.]{
        \includegraphics[width=0.45\textwidth]{00_TCC/01_LATEX/figuras/ch02_introducao_a_sde/estocastico_plot.png}
        \label{fig:ch02-exemplo-histograma-estocastico}
    }
    \caption{Comparação entre simulação determinística e estocástica do browniano acoplado.}
    \label{fig:ch02-exemplo-comparacao-simulacao}
\end{figure}

\begin{figure}[H]
    \centering
    \subfloat[Histograma do browniano acoplado determinístico.]{
        \includegraphics[width=0.45\textwidth]{00_TCC/01_LATEX/figuras/ch02_introducao_a_sde/hist_deterministico.png}
        \label{fig:ch02-exemplo-histograma-deterministico}
    }
    \hfill
    \subfloat[Histograma do browniano acoplado estocástico.]{
        \includegraphics[width=0.45\textwidth]{00_TCC/01_LATEX/figuras/ch02_introducao_a_sde/hist_estocastico.png}
        \label{fig:ch02-exemplo-histograma-estocastico}
    }
    \caption{Comparação entre histogramas das simulações determinística e estocástica do browniano acoplado.}
    \label{fig:ch02-exemplo-comparacao-histogramas}
\end{figure}

