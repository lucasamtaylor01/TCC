%!TeX root=../tese.tex
%("dica" para o editor de texto: este arquivo é parte de um documento maior)
% para saber mais: https://tex.stackexchange.com/q/78101

% As palavras-chave são obrigatórias, em português e em inglês, e devem ser
% definidas antes do resumo/abstract. Acrescente quantas forem necessárias.
\palavraschave{Palavra-chave1, Palavra-chave2, Palavra-chave3}

\keywords{Keyword1,Keyword2,Keyword3}

% O resumo é obrigatório, em português e inglês. Estes comandos também
% geram automaticamente a referência para o próprio documento, conforme
% as normas sugeridas da USP.
\resumo{
O presente trabalho tem como objetivo estudar a aproximação de sistemas dinâmicos rápidos-lentos por meio de equações diferenciais estocásticas, com ênfase no modelo de Lorenz 80. A análise inclui a formulação determinística do modelo, a introdução de incertezas através de ruídos estocásticos e a comparação entre simulações determinísticas e estocásticas. Como principal referência metodológica, este trabalho segue a abordagem proposta por \texorpdfstring{\citet{Chekroun2021}}{Chekroun et al., 2021}.
}


\abstract{
This undergratuated thesis aims to study the approximation of fast-slow dynamic systems using stochastic differential equations, with an emphasis on the Lorenz 80 model. The analysis includes the deterministic formulation of the model, the introduction of uncertainties through stochastic noise, and the comparison between deterministic and stochastic simulations. As its main methodological reference, this paper follows the approach proposed by \texorpdfstring{\citet{Chekroun2021}}{Chekroun et al., 2021}.
}
