\chapter{Introdução às equações diferenciais estocásticas}\label{cap:ch02-introducao-sde}

\newtheorem{definition}{Definição}[chapter]

Neste capítulo, faremos uma introdução às equações diferenciais estocásticas (EDEs).
Iniciaremos na Seção \ref{sec:ch02-motivacao-sde} com a apresentação de um problema motivador, que nos ajudará a entender a necessidade e a utilidade das EDEs. Na sequência, na Seção \ref{sec:ch02-consideracoes-estat}, reuniremos algumas definições estatísticas fundamentais que serão utilizadas ao longo do texto. 

Em seguida, na Seção \ref{sec:ch02-sde}, discutiremos os principais conceitos que compõem uma EDE e sua solução — como o movimento browniano (também conhecido como processo de Wiener) e a definição de integral estocástica. Também abordaremos, ainda que brevemente, o teorema de existência e unicidade para esse tipo de equação.

Por fim, com base em todas essas considerações, retornaremos ao problema motivador inicial e realizaremos uma simulação computacional utilizando a linguagem Julia, com foco na biblioteca SciML: Open Source Software for Scientific Machine Learning.

As principais referências utilizadas neste capítulo são: \citet{Evans2014} e \citet{Pavliotis2014}.

\section{Motivação} \label{sec:ch02-motivacao-sde}

Para iniciar a introdução às equações diferenciais estocásticas (EDEs), vamos apresentar a motivação para o seu estudo. Começamos com uma equação diferencial ordinária:

\begin{equation}
    \begin{cases}
        \dot{\mathbf{x}} = \mathbf{b}(\mathbf{x}(t))\\
        \mathbf{x}(0) = x_0, \label{eq:ch02_sistema_edo_generico}
    \end{cases}
\end{equation}

onde $t > 0$, $\mathbf{b}: \mathbb{R}^n \to \mathbb{R}^n$ é um vetor suave, e $\mathbf{x}(t): [0, \infty) \to \mathbb{R}^n$ representa a trajetória do sistema ao longo do tempo. O sistema \eqref{eq:ch02_sistema_edo_generico} é classificado como determinístico, pois, uma vez conhecida a condição inicial, a trajetória do sistema é totalmente determinada, conforme garante o Teorema de Existência e Unicidade.

Contudo, suponha que desejamos modelar a evolução de um fenômeno sujeito a incertezas. Em outras palavras, queremos aplicar o mesmo princípio das equações diferenciais ordinárias a um processo aleatório. Para isso, utilizamos as equações diferenciais estocásticas. Esse tipo de equação é definido como:

\begin{equation}
    \begin{cases}
        \dot{X}(t) = b(t, X(t)) + \sigma(t, X(t))\xi(t)\\
        \mathbf{X}(0) = x_0,
    \end{cases}
\end{equation}\label{eq:ch02_sistema_ede_generico}

onde $X(t) \in \mathbb{R}^d$, $b : [0,T] \times \mathbb{R}^d \mapsto \mathbb{R}^d$, e $\sigma : [0,T] \times \mathbb{R}^d \mapsto \mathbb{R}^{d \times m}$. Usamos a notação $\xi(t) = \frac{dW}{dt}$ para denotar (de forma formal) a derivada do movimento Browniano em $\mathbb{R}^m$, ou seja, um processo de ruído branco — um vetor Gaussiano generalizado, de média zero.

Como exemplo motivador para o uso de EDEs, consideremos a evolução do preço de uma ação na bolsa de valores ao longo do tempo. Esse preço é influenciado por diversos fatores externos, o que introduz um grau de incerteza quanto ao seu comportamento futuro. Assim, podemos modelá-lo por meio de uma EDE:

\begin{equation}
    \begin{cases}
        \dfrac{dS}{S} = \mu dt + \sigma dW\\
        \mathbf{S}(0) = s_0, \label{eq:ch02_exemplo_mercado_de_acoes}
    \end{cases}
\end{equation}

Nessa equação, $ \dfrac{dS}{S}$ representa a variação relativa do preço da ação ao longo do tempo; $\mu$ é uma constante que representa a taxa média de crescimento; e $\sigma$ expressa a volatilidade, ou seja, a incerteza associada aos diversos fatores que influenciam no preço.

Antes de continuar com esse exemplo, apresentaremos a base teórica necessária para a formulação da equação \eqref{eq:ch02_exemplo_mercado_de_acoes}, bem como suas propriedades.

\section{Considerações estatísticas}\label{sec:ch02-consideracoes-estat}
Primeiramente, é importante destacar que parte das definições apresentadas nesta seção fazem uso da teoria da medida no contexto da teoria das probabilidades. Embora esses conceitos não sejam abordados neste trabalho, o leitor interessado em se aprofundar no tema pode recorrer a \citet{Evans2014}, que oferece uma excelente (e rápida) introdução ao assunto.

\begin{definition}[Processo Estocástico]\label{def:ch02-processo-estocastico}
    Uma coleção $\{X(t) \mid t \geq 0\}$ de variáveis aleatórias é chamada de \textit{processo estocástico}.
\end{definition}

\begin{definition}[Filtração] \label{def:ch02-filtracao}
    Seja $W(\cdot)$ um processo estocástico. Denotamos por $\mathcal{W}(t) := \sigma(W(s) : 0 \leq s \leq t)$ a $\sigma$-álgebra gerada por $W$ até o tempo $t$, isto é, o conjunto de todos os eventos observáveis do processo até esse instante.

    Dizemos que uma família $\mathcal{F}(\cdot)$ de $\sigma$-álgebras $\subseteq \mathcal{U}$ é uma \textit{filtração} (com respeito a $W(\cdot)$) se satisfaz as seguintes propriedades:
    
    \begin{enumerate}
        \item $\mathcal{F}(t) \supseteq \mathcal{F}(s)$ para todo $t \geq s \geq 0$;
        \item $\mathcal{F}(t) \supseteq \mathcal{W}(t)$ para todo $t \geq 0$;
        \item $\mathcal{F}(t)$ é independente de $\mathcal{W}^{+}(t) := \sigma(W(u) - W(t) : u > t)$ para todo $t \geq 0$.
    \end{enumerate}
\end{definition}

\begin{definition}[$\mathcal{F}$-mensurabilidade]\label{def:ch02-f-mensuravel}
    Seja $(\Omega, \mathcal{F}, P)$ um espaço de probabilidade. Dizemos que uma função 
    \begin{equation*}
        X : \Omega \to \mathbb{R}^n 
    \end{equation*}
    é $\mathcal{F}$-mensurável se, para todo boreliano $B \subseteq \mathbb{R}^n$, a pré-imagem 
    \begin{equation*}
        X^{-1}(B) = \{\omega \in \Omega : X(\omega) \in B\}
    \end{equation*}
    pertence a $\mathcal{F}$.
\end{definition}


\begin{definition}[\textit{Martingale}] \label{def:ch02-martingale}
    Seja $\left\{ \mathcal{F}_t \right\}_{t \in [0,T]}$ uma filtração definida no espaço de probabilidade $(\Omega, \mathcal{F}, \mu)$, e seja $\left\{ \mathcal{M}_t \right\}_{t \in [0,T]}$ um processo adaptado a $\mathcal{F}_t$, com $\mathcal{M}_t \in L^1(0,T)$. Dizemos que $\mathcal{M}_t$ é uma martingala em relação a $\mathcal{F}_t$ se
\begin{equation*}
    \mathbb{E}[\mathcal{M}_t \mid \mathcal{F}_s] = \mathcal{M}_s \quad \forall t \geq s.
\end{equation*}
\end{definition}
\section{Equações diferenciais estocásticas}\label{sec:ch02-sde}

Dadas as considerações estatísticas, passemos para a construção da equação \eqref{eq:ch02_sistema_ede_generico}, reproduzida novamente a seguir:
\begin{equation*}
    \begin{cases}
        \dot{X}(t) = b(t, X(t)) + \sigma(t, X(t))\,\xi(t),\\
        X(0) = x_0, \label{eq:ch02_sistema_ede_generico_2}
    \end{cases}
\end{equation*}

A solução $X(\cdot)$ é então expressa como:
\begin{equation}
    X(t) = x_0 + \int_0^t b(s, X(s))\,ds + \int_0^t \sigma(s, X(s))\,dW_s. \label{eq:ch02-solucao-integral}
\end{equation}

A equação \eqref{eq:ch02-solucao-integral} possui uma solução bem definida, garantida pelo Teorema de Existência e Unicidade para equações diferenciais estocásticas, cuja demonstração completa pode ser consultada em \citet{Evans2014}. Esse resultado possui papel fundamental, assim como no caso das equações diferenciais ordinárias, pois assegura não apenas que uma solução existe, mas que ela é única dadas as condições iniciais especificadas. Tal garantia é essencial para que a modelagem estocástica tenha validade teórica e aplicação prática.

Para que o sistema \eqref{eq:ch02_sistema_ede_generico_2} esteja completamente definido, ainda é necessário esclarecer dois elementos fundamentais: a natureza do ruído $\xi(t)$ e a interpretação da integral estocástica $\int_0^t f(s, X(s)),dW_s$. Nesta seção, abordaremos em detalhes esses dois conceitos, explorando suas definições formais e principais propriedades, a fim de justificar rigorosamente a formulação do modelo estocástico apresentado.

\subsection{Processo de Wiener} \label{subsec:ch02_processo_de_wiener}

\subsubsection{Desenvolvimento histórico} \label{subsubsection:ch02-desenvolvimento-hist-wiener}
Um dos elementos fundamentais para o estudo de equações diferenciais estocásticas é o \textit{movimento browniano}. Trata-se de um fenômeno físico que envolve o estudo do movimento de grãos de pólen suspensos na água.

Historicamente, o problema foi inicialmente proposto e estudado por Robert Brown entre 1826 e 1827. Em seus estudos, Brown observou que a trajetória das partículas era irregular e que o movimento de duas partículas distintas parecia ser independente. Em 1905, Albert Einstein retomou o problema, relacionando-o à equação da difusão\footnote{Detalhes em \citet{Evans2014}}.

Finalmente, na década de 1920, o fenômeno foi formalizado matematicamente por Norbert Wiener. Essa formalização é crucial para a definição rigorosa do movimento browniano, tanto que outra forma de referir-se a ele é \textit{processo de Wiener} — forma que utilizaremos a fim de diferenciar do problema físico original.

\subsubsection{Definição e propriedades}\label{subsubsection:definicao-e-propriedades}

\begin{definition}
    Um processo estocástico real, denotado por $W(\cdot)$ é chamado de \textit{processo de Wiener} padrão quando satisfaz as seguintes propriedades:
    \begin{enumerate}
        \item $W(0) = 0$, quase certamente;
        \item Para todo $t \geq s \geq 0$, tem-se que $W(t) - W(s) \sim \text{Normal}(0,t-s)$;
        \item $W(\cdot)$ possui incrementos independentes, isto é, para $0 < t_1 < t_2 < \cdots < t_n$, as variáveis aleatórias
        \begin{equation*}
            W(t_1), \; W(t_2) - W(t_1), \; \ldots, \; W(t_n) - W(t_{n-1}) \text{  são independentes.}
        \end{equation*}
    \end{enumerate}
    \hfill \citep{Evans2014} 
\end{definition}


A função densidade de probabilidade do processo de Wiener padrão unidimensional, definido como $W(t): \mathbb{R}^+ \to \mathbb{R}$, é dada por:
\begin{equation*} 
f(x; t, s) = \left( 2\pi (t - s) \right)^{-\tfrac{1}{2}} \exp\!\left( -\frac{x^{2}}{2(t - s)} \right). 
\end{equation*}

No caso do processo de Wiener padrão $n$-dimensional, $W(t): \mathbb{R}^+ \to \mathbb{R}^n$, onde cada componente $W_i(t)$, com $i = 1, \ldots, n$, é um processo de Wiener unidimensional independente, a função densidade do vetor gaussiano aleatório $W(t) - W(s)$ é dada por:
\begin{equation*} 
g(\mathbf{x}; t, s) = \left( 2\pi (t - s) \right)^{-n/2} \exp\!\left( -\frac{\|\mathbf{x}\|^2}{2(t - s)} \right). 
\end{equation*}

As principais propriedades do processo de Wiener padrão são:
\begin{enumerate}
    \item \textbf{Reescalonamento.} Para cada $c > 0$, defina $X_t = \tfrac{1}{\sqrt{c}} W(ct)$. Então $(X_t,\, t \geq 0) = (W_t,\, t \geq 0)$.

    \item \textbf{Translação.} Para cada $c > 0$, $W_{c+t} - W_c$, $t \geq 0$, é um processo de Wiener que é independente de $W_u$, $u \in [0,c]$.

    \item \textbf{Reversão no tempo.} Defina $X_t = W_{1-t} - W_1$, $t \in [0,1]$. Então $(X_t,\, t \in [0,1]) = (W_t,\, t \in [0,1])$.

    \item \textbf{Inversão.} Seja $X_t, t \geq 0$, definido por $X_0 = 0$, $X_t = tW(1/t)$. Então $(X_t,\, t \geq 0) = (W_t,\, t \geq 0)$.
\end{enumerate}
\hfill \citep{Pavliotis2014}

\subsection{Integral estocástica} \label{subsec:ch02-integral-estocastica}
Nesta seção, vamos definir uma integral estocástica, representada genericamente abaixo:
\begin{equation}
    I(t) = \int_0^t f(s) \;dW_s. \label{eq:ch02-integral-estocastica-generica}
\end{equation}

Para definir o que significa uma integral estocástica, seguiremos a abordagem de \citet{Evans2014}. A partir dela, definiremos a integral de Itô e suas propriedades, além de mencionar a integral de Stratonovich conforme apresentada por \citet{Pavliotis2014}.

Inicialmente, é importante destacar que \citet{Evans2014} introduz a definição de $\int_0^t f(\cdot) \, dW$ em três etapas: primeiro, quando $f(\cdot)$ é uma função determinística simples (constante por partes); em seguida, quando $f(\cdot)$ é determinística geral em $L^2([0,T])$; e, por fim, quando $f(\cdot)$ é um processo estocástico adaptado mais geral. 

Embora todos os casos estejam bem definidos, focaremos apenas neste último, aquele em que $f(\cdot)$ é um processo estocástico adaptado, por ser o mais relevante para os nossos objetivos. Essa escolha também se justifica pelo fato de \citet{Pavliotis2014} adotar abordagem semelhante, ainda que de forma menos detalhada.

Assumiremos, portanto, que $f$ pertence a $L^2([0,t]\times \Omega)$, isto é:
\begin{equation}
    \mathbb{E}\left(\int_0^t f(s)^2 \; ds\right) < \infty. \label{eq:ch02-processo-estocastico-l2}
\end{equation}

Além disso, assumimos que o integrando $f(t)$ é $\mathcal{F}_t$-mensurável, onde $\mathcal{F}_t$ é a filtragem gerada pelo movimento browniano $W(t)$. No nosso contexto, a partir da definição \ref{def:ch02-filtracao}, isso significa que $f$ depende apenas da informação disponível até o instante $t$, ou, de forma equivalente, não depende do futuro.

Dadas essas considerações em relação à função $f$, voltemos à equação \eqref{eq:ch02-integral-estocastica-generica}. Primeiro, tomamos um intervalo $[0,T]$, sem seguida, definimos uma partição $P$ deste intervalo:
\begin{equation*}
        P := \{0 = t_0 < t_1 < \cdots < t_m = T\}.
\end{equation*}
    
Para $0 \leq \lambda \leq 1$ fixado e $P$ uma partição de $[0,T]$, definimos
\begin{equation}
    \tau_k := (1-\lambda)t_k + \lambda t_{k+1}  \quad k = 0,\dots,m-1. \label{ch02-particao-tau}
\end{equation}

Para a partição $P$ e para $0 \leq \lambda \leq 1$, definimos
\begin{equation*}
    \sum_{k=0}^{m-1} f(\tau_k)\left(W(t_{k+1}) - W(t_k)\right).
\end{equation*}

A partir do conceito de aproximação de Riemman, tomamos $m \to \infty$ e podemos definir a integral $I(t)$ da seguinte forma:
\begin{equation}
    I(t) = \lim_{m \to \infty}  \sum_{k=0}^{m-1} f(\tau_k)\left(W(t_{k+1}) - W(t_k)\right). \label{eq:ch02-approx-riemman}
\end{equation}

Temos que a definição de \eqref{eq:ch02-approx-riemman} depende da escolha de $\lambda \in [0,1]$ em \eqref{ch02-particao-tau}. Se $\lambda = 0$, obtemos a integral estocástica de Itô:
\begin{equation*}
    I_I(t) := \lim_{K \to \infty} \sum_{k=0}^{K-1}  f(t_k) \left( W(t_{k+1}) - W(t_k) \right) = \int_0^t f(s) \; dW_s.
\end{equation*}
Se tomarmos $\lambda = \frac{1}{2}$. definimos a integral estocástica de Stratonovich:
\begin{equation*}
    I_S(t) := \lim_{K \to \infty} \sum_{k=0}^{K-1} 
    f\left(\frac{1}{2}(t_k + t_{k+1})\right)\left(W(t_{k+1}) - W(t_k)\right) = \int_0^t f(s) \; \circ  dW_s.
\end{equation*}

Ambas são duas formas diferentes de abordar a mesma questão, porém cada uma contém suas vantagens. Em relação a forma de Itô, temos que
\begin{enumerate}
    \item Valem as seguintes fórmulas:
    \begin{align*}
        \mathbb{E}\left(\int_0^t f(s) \; dW_s\right) &=0\\
        \mathbb{E}\left(\left( \int_o^t f(s) \; dW_s \right)^2 \right) &= \mathbb{E}\left(\int_0^t f^2(s) \; ds \right)
    \end{align*}
    \item $I_s$ é um \textit{marigale} (vide definição \ref{def:ch02-martingale})
\end{enumerate}
\hfill \citep{Evans2014}

Em relação a forma de Stratoinovich, temos as seguintes vantagens:
\begin{enumerate}
    \item A regra da cadeia usual é válida.
    \item As soluções de EDEs são estáveis em relação a mudanças nos termos aleatórios.
\end{enumerate}
\hfill \citep{Evans2014}

\section{Simulação de uma equação diferencial estocásticas} \label{ch02-simulacoes}

Dadas as considerações teóricas, voltaremos ao problema motivador.
\begin{equation*}
    \begin{cases}
        \dfrac{dS}{S} = \mu dt + \sigma dW\\
        \mathbf{S}(0) = s_0,
    \end{cases}
\end{equation*}

Aplicando a regra da cadeia de Itô, temos que a solução do problema é dada por:
\begin{equation*}
    S(t) = s_0 \cdot \exp \left( \sigma W(t) + \left( \mu - \frac{\sigma^2}{2}t \right)\right)
\end{equation*}

Dada toda teoria exposta, sabemos que o resultado está bem definido. Mais que isso, a partir dele podemos realizar uma simulação computacional utilizando a biblioteca \textit{SciML: Open Source Software for Scientific Machine Learning} \citep{Rackauckas2017}. O código-fonte correspondente pode ser consultado no Apêndice \ref{app_sec:mercado-de-acoes}.

Para a simulação, adotamos os seguintes parâmetros: $\mu = 0{,}05$, representando uma taxa de crescimento anual; $\sigma = 1{,}0$, quantificando a volatilidade do mercado no período considerado; e um passo de tempo de $\frac{1}{365 \cdot 24}$, o que corresponde à variação por hora dos preços. Estabelecemos ainda o preço inicial da ação em R\$~12{,}00. Com base nesses parâmetros, obtivemos o gráfico a seguir:

\begin{figure}[H]
    \centering
    \includegraphics[width=0.75\linewidth]{00_TCC/01_LATEX/figuras/ch02_intro_sde/stock_market_evolution.png}
    \caption{Simulação do preço de uma ação a partir da EDE \eqref{eq:ch02_exemplo_mercado_de_acoes}}
    \label{fig:ch02-simulacao-mercado-de-acoes}
\end{figure}

Vale destacar que: por se tratar de um fenômeno aleatório, temos que diferentes simulações geram diferentes resultados. Contudo, nas simualações computacionais, é possível definir o valor de uma \textit{seed} garantindo reprodutibilidade do código.

