\chapter{Aproximação de sistemas determinsiticos} \label{cap:approx-sis-det}

\section{Exemplo simplificado}

A seguir, apresentemos um exemplo retirado de \citet{Pavliotis2008} de uma aproximação de um sistema determinístico para um sistema estocástico. Tal exemplo é relevante, pois trata-se de uma abordagem simplificada do que vamos realizar com o modelo de Lorenz 80.

O exemplo trata-se de um movimento browniano, assim como apresentado em na seção \ref{subsec:ch02-propriedades-movimento-browniano} acoplado ao sistema Lorenz 63 apresentado em \ref{eq:ch01-lorenz63} a partir das variáveis $y=(y_1, y_2, y_3)^\intercal$. O exemplo em questão é expresso por:

\begin{equation}\label{eq:ch02_exemplo_de_aprox}
    \begin{aligned}
        \frac{dx}{dt} &= x - x^{3} + \frac{\lambda}{\varepsilon} y_{2}, \\
        \frac{dy_{1}}{dt} &= \frac{10}{\varepsilon^{2}} (y_{2} - y_{1}), \\
        \frac{dy_{2}}{dt} &= \frac{1}{\varepsilon^{2}} \left( 28 y_{1} - y_{2} - y_{1} y_{3} \right), \\
        \frac{dy_{3}}{dt} &= \frac{1}{\varepsilon^{2}} \left( y_{1} y_{2} - \tfrac{8}{3} y_{3} \right).
    \end{aligned}
\end{equation}

\textbf{EXEMPLICAR POR QUE PODEMOS APROXIMAR}

Podemos aproximar o modelo para sua versão estocástica forma de Itô:
\begin{equation}
    \frac{dX}{dt} = X - X^{3} + \sigma \frac{dW}{dt},
\end{equation}

Onde $\sigma$ é expresso por:
\begin{equation}
    \sigma^{2} = 2 \lambda^{2} \int_{0}^{\infty} \frac{1}{T}  \left( \lim_{T \to \infty} \int_{0}^{T} \psi^{s}(y)\, \psi^{t+s}(y)\, ds \right) dt.
\end{equation}

A partir das equações apresentadas, podemos realizar simulações computacionais. Novamente, as simulações foram realizadas com o uso da biblioteca \textit{SciML} \citep{Rackauckas2017} e os programas que geraram os dados estão no apêndice \ref{app:apendice-lista-de-programas}. 


\begin{figure}[H]
	\centering
	\begin{subfigure}{0.48\linewidth}
		\centering
		\includegraphics[width=\linewidth]{00_TCC/01_LATEX/figuras/ch03_approx_com_sde/evolucao_x_deterministico.png}
		\caption{Simulação de 1 dia com $f_1=0.1$}
		\label{fig:ch02-evolucao-x-deterministico}
	\end{subfigure}
	\hfill
	\begin{subfigure}{0.48\linewidth}
		\centering
		\includegraphics[width=\linewidth]{00_TCC/01_LATEX/figuras/ch03_approx_com_sde/evolucao_x_estocastico.png}
		\caption{Simulação de 50 dias com $f_1=0.1$}
		\label{fig:fig:ch02-evolucao-x-estocastico}
	\end{subfigure}
	\caption{Evolução temporal das variáveis $x_1$, $y_1$ e $z_1$ para $f_1=0.1$. À esquerda, simulação curta (1 dia). À direita, simulação longa (50 dias).}
	\label{fig:ch02-exemplo-evolucao-x}
\end{figure}

\begin{figure}[H]
    \centering
    
    \begin{subfigure}{0.48\linewidth}
        \centering
        \includegraphics[width=\linewidth]{00_TCC/01_LATEX/figuras/ch03_approx_com_sde/hist_deterministico.png}
        \caption{Simulação de 50 dias com $f_1=0.1$.}
        \label{fig:ch02-histograma-deterministico}
    \end{subfigure}
    \hfill
    \begin{subfigure}{0.48\linewidth}
        \centering
        \includegraphics[width=\linewidth]{00_TCC/01_LATEX/figuras/ch03_approx_com_sde/hist_estocastico.png}
        \caption{Simulação de 1 dia com $f_1=0.1$.}
        \label{fig:ch02-histograma-estocastico}
    \end{subfigure}

    \vspace{0.5cm}
    \begin{subfigure}{0.65\linewidth}
        \centering
        \includegraphics[width=\linewidth]{00_TCC/01_LATEX/figuras/ch03_approx_com_sde/hist_comparacao_densidade.png}
        \caption{Comparação das distribuições empíricas: determinística vs. estocástica.}
        \label{fig:ch02-histograma-sobreposto}
    \end{subfigure}
    
    \caption{Histogramas e comparações das simulações para $f_1=0.1$. Acima, histogramas individuais: à esquerda, simulação curta (1 dia); à direita, simulação longa (50 dias). Em destaque (abaixo), a sobreposição das densidades.}
    \label{fig:ch02-exemplo-histogramas}
\end{figure}
