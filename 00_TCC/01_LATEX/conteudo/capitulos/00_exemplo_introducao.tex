%!TeX root=../tese.tex
%("dica" para o editor de texto: este arquivo é parte de um documento maior)
% para saber mais: https://tex.stackexchange.com/q/78101

%% ------------------------------------------------------------------------- %%

% "\chapter" cria um capítulo com número e o coloca no sumário; "\chapter*"
% cria um capítulo sem número e não o coloca no sumário. A introdução não
% deve ser numerada, mas deve aparecer no sumário. Por conta disso, este
% modelo define o comando "\chapter**".
\chapter**{Introdução}\label{cap:introducao}

\enlargethispage{.5\baselineskip}
Explicar geral do trabalho e motivações

Com o objetivo de organizar o desenvolvimento das ideias e facilitar a leitura, este trabalho está dividido em três capítulos e \textbf{três apêndices}.

O capítulo \ref{cap:ch01-lorenz-deterministico} apresenta o modelo de Lorenz 80 em sua formulação determinística. Nele, discutimos a motivação por trás do modelo, sua construção e simulações, com base no artigo original de \citet{Lorenz1980}.

No capítulo \ref{cap:ch02-introducao-sde}, introduzimos o conceito de equações diferenciais estocásticas, abordando os fundamentos matemáticos necessários para sua formulação, além de propriedades teóricas e exemplos ilustrativos.

O capítulo \ref{cap:approx-sis-det} constitui o núcleo deste trabalho, onde realizamos simulações voltadas à aproximação de sistemas determinísticos por meio de ruído estocástico. Essa abordagem é aplicada ao modelo de Lorenz 80, agora sob a abordagem estocástica introduzida por \citet{Chekroun2021}.

Complementam o texto \textbf{três apêndices}: no apêndice \ref{app:apendice-consideracoes-matematicas}, reunimos considerações matemáticas utilizadas ao longo do trabalho; no apêndice \ref{app:apendice-lista-de-programas}, disponibilizamos o código-fonte das simulações; e, por fim, o apêndice \ref{app:apendice-mori-zwanzig} trata da construção e aplicação do formalismo de Mori-Zwanzig.



Por fim, conforme as boas práticas no uso responsável de inteligência artificial, o código utilizado neste trabalho foi produzido com auxílio da ferramenta \textit{GitHub Copilot} e o texto foi aperfeiçoado com \textit{Clarice.ai}.