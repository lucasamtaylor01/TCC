\chapter{Formalismo Mori-Zwanzig} \label{app:apendice-mori-zwanzig}

\section{Introdução}\label{sec:ch03-introducao-mz}
Neste capítulo, apresentaremos o formalismo de Mori-Zwanzig (MZ). Para isso iniciamos com uma breve discussão sobre o formalismo MZ e sua importância no contexto deste trabalho (seção \ref{sec:ch03-motivacao-mz}). Em seguida, revisamos conceitos matemáticos fundamentais para sua formulação, como o operador de Liouville, a notação de semigrupos e os polinômios hermitianos (seção \ref{sec:ch03-preliminares-mz}). Por fim, na seção \ref{sec:ch03-mz}, desenvolvemos o formalismo passo a passo, analisando cada termo de forma detalhada.

Como referência utilizamos o trabalho do matemático Alexandre J. Chorin: \citet{Chorin2000}, \citet{Chorin2002} e, principalmente, \citet{Chorin2013}. 

\section{Motivação} \label{sec:ch03-motivacao-mz}
O método de Mori-Zwanzig, desenvolvido por Robert Walter Zwanzig e Hajime Mori na segunda metade do século XX, é utilizado em sistemas hamiltonianos. Esse método consiste em classificar as variáveis do sistema em duas categorias: ``resolvidas'' e ``não resolvidas''. As variáveis resolvidas são aquelas cujos comportamentos e valores são bem conhecidos, enquanto as não resolvidas são aquelas para as quais não se possui informações diretas. Para substituir essas variáveis não resolvidas, o método introduz termos estocásticos, denominados ruídos (\textit{noise}), além de um termo de amortecimento (\textit{damping}), também conhecido como termo de memória (\textit{memory term}). Essa abordagem permite que o comportamento do sistema de interesse seja preservado de maneira adequada, mesmo sem conhecer completamente as variáveis não resolvidas.

O principal objetivo do artigo de \citet{Chekroun2021} é simplificar o modelo de Lorenz 80, preservando seu comportamento. Para isso, utilizaremos o método de Mori-Zwanzig, que é uma abordagem física-estatística aplicável em sistemas como o L80. Dada a relevância deste método para o trabalho de \citet{Chekroun2021}, optamos por incluir uma introdução ao formalismo de Mori-Zwanzig, a fim de proporcionar uma melhor compreensão de sua aplicação no contexto do modelo de Lorenz 80 e permitir uma base teórica sólida para eventuais explorações.

\section{Preliminares} \label{sec:ch03-preliminares-mz}
\subsection{Convertendo sistemas de EDO não lineares como sistemas de EDPs lineares} \label{subsec:ch03-edo-naolin-edplin}

Considere o sistema de equações diferenciais ordinárias (EDO) dado por:
\begin{equation}
	\frac{d}{dt} \varphi(x,t) = R(\varphi(x,t)), \quad \varphi(x, 0) = x,
	\label{eq:ch03-sistema-edo}
\end{equation}

onde $R$ é uma função não linear, $\varphi$ é uma função dependente do tempo, e $R$, $\varphi$ e $x$ podem assumir dimensões infinitas, sendo formados pelos vetores $R_i$, $\varphi_i$ e $x_i$, respectivamente.

A partir disso, podemos definir o \textit{Operador de Liouville} associado à equação \eqref{eq:ch03-sistema-edo} como:
\begin{equation}
	L = \sum_i R_i(x) \frac{\partial}{\partial x_i}
	\label{eq:ch03-operator-liouville}
\end{equation}

Utilizando o \textit{Operador de Liouville}, podemos transformar o sistema de EDOs não lineares em um sistema de equações diferenciais parciais (EDPs) lineares da forma:
\begin{equation}
	u_t = Lu, \quad u(x,0) = g(x)
	\label{eq:ch03-sistema-edp-linear}
\end{equation}

A solução desse sistema existe, é única, e é dada por:
\begin{equation}
	u(x,t) = g(\varphi(x,t))   
	\label{eq:ch03-solucao-sistema-edp-linear}
\end{equation}
Portanto, temos que a equação \eqref{eq:ch03-sistema-edp-linear} é bem definida\footnote{Detalhes da demonstração podem ser encontrados em \citet[p.~181-182]{Chorin2013}}.

\subsection{Notação de semigrupo} \label{subsec:ch03-semigrupo}

Tomemos $X$ um conjunto não vazio, dotado de uma operação binária $ * $, ou seja, $ X \times X \to X $, que satisfaz a propriedade de associatividade:
\begin{equation*}
	(a * b) * c = a * (b * c), \quad \forall a,b,c \in X.
\end{equation*}

A notação de semigrupo oferece uma forma compacta e eficiente de representar soluções para equações diferenciais, particularmente as parciais ou de evolução.

Considere o operador $\Delta$ definido por:
\begin{equation*}
	\Delta \psi = \psi_{xx}, \quad \text{onde $\psi$ é uma função suave}.
\end{equation*}
Agora, considere a equação diferencial:
\begin{equation*}
	\frac{dv}{dt} - kv = 0, \quad v(0) = v_0,
\end{equation*}
cuja solução é bem conhecida: $ v(t) = v_0 e^{kt} $.

De forma análoga, considere a equação do calor:
\begin{equation*}
	v_t - \frac{1}{2} \Delta v = 0, \quad v(x, 0) = \varphi(x),
\end{equation*}
onde $v_t$ é a derivada de $v$ em relação ao tempo e $\varphi(x)$ é a condição inicial. Em vez de resolver diretamente, expressamos a solução utilizando a notação de semigrupo:
\begin{equation*}
	v(t) = e^{\frac{1}{2}t \Delta} \varphi.
\end{equation*}
Aqui, $\mathbb{E}^{\frac{1}{2}t \Delta}$ é um operador semigrupo gerado pela operação de difusão (pelo operador $\Delta$). Ele é aplicado à condição inicial $\varphi(x)$, e a solução $v(t)$ descreve a evolução temporal de $v(x,t)$ ao longo do tempo $t$. Essa notação permite representar soluções de equações diferenciais de maneira compacta, explorando a estrutura associativa da operação de semigrupo. Especificamente, ela satisfaz a propriedade de composição:
\begin{equation*}
	e^{\frac{1}{2}(t+s)\Delta} = e^{\frac{1}{2}t \Delta} e^{\frac{1}{2}s \Delta}.
\end{equation*}

Dada a notação de semigrupo apresentada anteriormente, aplicamos esta notação à equação \eqref{eq:ch03-solucao-sistema-edp-linear}:
\begin{equation}
	e^{tL}g(x) = g(\varphi(x,t))
	\label{eq:ch03-solucao-sistema-edp-linear-semigrupo}
\end{equation}

Note que $\mathbb{E}^{tL}x$ não representa uma avaliação direta de $\mathbb{E}^{tL}$, mas sim a ação do operador $\mathbb{E}^{tL}$ sobre o vetor formado pelos componentes $x_i$. Além disso, a função $g$ comuta com a variação temporal das condições iniciais de $x_i$. 

Vale destacar que $g$ é uma função independente do tempo em relação às variáveis que descrevem o sistema físico, e sua variação ocorre exclusivamente devido à mudança dessas variáveis ao longo do tempo. Assim, a equação \eqref{eq:ch03-sistema-edp-linear} pode ser expressa como:
\begin{equation}
	Le^{tL} = e^{tL}L
\end{equation}

Essa mesma relação se aplica a matrizes: sejam $A$ e $B$ duas matrizes, então a seguinte identidade é válida:
\begin{equation}
	\exp(t(A+B)) = \exp(tA) + \int_0^t \exp\left((t-s)(A+B)\right)B\exp(sA) \, ds
	\label{eq:ch03-formula-de-duhamel}
\end{equation}

Esta equação, conhecida como \textit{Fórmula de Duhamel} ou \textit{fórmula de Dyson}, é bem definida.

\subsection{Polinômios Hermitianos} \label{subsec:ch03-hermitianos}

Primeiramente, definimos o produto interno como:
\begin{equation}
	\langle u, v \rangle = \int_{-\infty}^{+\infty} \frac{e^{-x^2/2}}{\sqrt{2 \pi}} u(x) v(x) \, dx
	\label{eq:ch03-produto-interno-hermitiano-unidimensional}
\end{equation}

Os polinômios $p_n(x)$ e $p_m(x)$ são ortonormais em relação a esse produto interno \eqref{eq:ch03-produto-interno-hermitiano-unidimensional} quando satisfazem a seguinte condição:
\begin{equation*}
	\langle p_n, p_m \rangle = \int_{-\infty}^{\infty} e^{-x^2/2} p_n(x) p_m(x) \, dx = \delta_{nm},
\end{equation*}
em que $\delta_{nm}$ é o delta de Kronecker, que apresenta as propriedades:

\begin{enumerate}
	\item \textbf{Ortogonalidade}: Para $n \neq m$, os polinômios são ortogonais, ou seja, o produto interno entre eles é nulo:
	      \begin{equation*}
	      	\langle p_n, p_m \rangle = 0 \quad \text{quando} \quad n \neq m
	      \end{equation*}
	      	          
	\item \textbf{Normalização}: Para $n = m$, os polinômios são normalizados, de modo que o produto interno é igual a 1:
	      \begin{equation*}
	      	\langle p_n, p_n \rangle = 1
	      \end{equation*}
\end{enumerate}

No caso $n$-dimensional, o produto interno se generaliza para:
\begin{equation*}
	\langle u, v \rangle = \int_{-\infty}^{+\infty} \cdots \int_{-\infty}^{+\infty} (2 \pi)^{-n/2} \exp \left(-\sum_{i=1}^n \frac{x_i^2}{2} \right) u(x) v(x) \, dx_1 \ldots dx_n
\end{equation*}

De forma mais geral, se $H(q, p)$ é um Hamiltoniano, é possível definir uma família de polinômios nas variáveis $q$ e $p$ que sejam ortonormais com respeito à densidade canônica $Z^{-1} e^{-H/T}$. Os polinômios que satisfazem essa condição ainda são chamados de \textit{polinômios hermitianos}.

Por fim, para o formalismo de Mori-Zwanzig, consideraremos um espaço $n$-dimensional $\Gamma$ com uma densidade de probabilidade dada. Dividiremos as coordenadas em dois tipos: $\hat{x}$ e $\tilde{x}$. Seja $g$ uma função de $x$; então $\mathbb{P}g = \mathbb{E}[g \mid \hat{x}]$ é uma projeção ortogonal sobre o subespaço das funções de $\hat{x}$.Temos que essa projeção gera um subespaço de polinômios hermitianos que são funções de $\hat{x}$ e projetando sobre esses polinômios.

\section{Mori-Zwanzig} \label{sec:ch03-mz}
\subsection{Construção} \label{sec:ch03-construcao-mz}
Tomemos novamente o sistema \eqref{eq:ch03-sistema-edo}, reproduzido abaixo:
\begin{equation*}
	\frac{d}{dt} \varphi(x,t) = R(\varphi(x,t)), \quad \varphi(x, 0) = x,
\end{equation*}

Lembremos que a equação é composta por componentes de dimensão $n$. Dentre essas $n$ componentes, definimos as primeiras $m$ componentes de $\varphi$, com $m < n$, como as variáveis de interesse. Em seguida, classificamos $\hat{\varphi}$ como as variáveis ``resolvidas'' e $\tilde{\varphi}$ como as variáveis ``não resolvidas'':
\begin{equation*}
	\varphi = (\hat{\varphi}, \tilde{\varphi}), \quad \hat{\varphi} = \left(\varphi_1, \ldots, \varphi_m \right), \quad \tilde{\varphi} = \left(\varphi_{m+1}, \ldots, \varphi_n\right)
\end{equation*}
O mesmo vale para $x$ e $R$: $x = (\hat{x}, \tilde{x})$ e $R = (\hat{R}, \tilde{R})$. A partir das variáveis resolvidas, buscamos criar predições para o modelo de interesse, utilizando as soluções de uma parte da equação.

Com base no \textit{Operador de Liouville} e na \textit{notação de semigrupo}, podemos reescrever as componentes de $\hat{\varphi}$ como\footnote{Note que cada componente depende de \textbf{todos} os valores de $x$. Portanto, se $\tilde{x}$ for aleatório, então $\hat{\varphi}$ também será.}:
\begin{equation*}
	\hat{\varphi}_j(x,t) = e^{tL}x_j, \quad 1 \leq j \leq m
\end{equation*}
Ainda na notação de semigrupo, a equação dessas componentes é dada por:
\begin{equation}
	\frac{\partial}{\partial t}e^{tL}x_j = Le^{tL}x_j = e^{tL}Lx
	\label{eq:ch03-mori-zwanzig-notacao-semigrupo}
\end{equation}

A partir da \textit{projeção ortogonal} introduzida na seção anterior, definimos $\mathbb{P}$ como a projeção dada por: $\mathbb{P}g(x) = \mathbb{E}[g|\hat{x}]$. Assumimos que, no instante $t = 0$, conhecemos a densidade conjunta de todas as variáveis $x$, mas apenas os dados iniciais $\hat{x}$ são conhecidos. A densidade das variáveis em $\tilde{x}$ é, então, a densidade conjunta de todas as variáveis $x$ com $\hat{x}$ fixado. Assim, $\mathbb{P}$ é uma projeção sobre um espaço de funções com variáveis fixas e, portanto, independente do tempo.

As projeções $\mathbb{P}\hat{\varphi}(t) = \mathbb{E}[\hat{\varphi}(t) | \hat{x}]$ são de nosso maior interesse, pois estimam o comportamento do sistema a partir de um conjunto reduzido de variáveis.

Definindo $\mathbb{Q} = I - \mathbb{P}$ e considerando que as seguintes propriedades são válidas para quaisquer projeções ortogonais:
\begin{enumerate}
	\item $\mathbb{P}^2 = \mathbb{P}$;
	\item $\mathbb{Q}^2 = \mathbb{Q}$;
	\item $\mathbb{P}\mathbb{Q} = 0$.
\end{enumerate}

Podemos reescrever a equação \eqref{eq:ch03-mori-zwanzig-notacao-semigrupo} como:
\begin{equation}
	\frac{\partial }{\partial t}e^{tL}x_j = e^{tL}\mathbb{P}Lx_j + e^{tL}\mathbb{Q}Lx_j
	\label{eq:ch03-mori-zwanzig-pre-mz}
\end{equation}

Utilizando agora a \textit{fórmula de Dyson}, com $A = \mathbb{Q}L$ e $B = \mathbb{Q}L$, obtemos:
\begin{equation}
	e^{tL} = e^{t\mathbb{Q}L} + \int_0^t e^{(t-s)L} \mathbb{P}L e^{s\mathbb{Q}L} \, ds
	\label{eq:mori-zwanzig-dyson-mz}
\end{equation}

Pela linearidade da equação de Liouville e a partir das equações \eqref{eq:ch03-mori-zwanzig-pre-mz} e \eqref{eq:mori-zwanzig-dyson-mz}, obtemos:
\begin{equation}
	\frac{\partial}{\partial t} e^{tL} x_j = e^{tL} \mathbb{P}L x_j + e^{t\mathbb{Q}L} \mathbb{Q}L x_j + \int_0^t e^{(t-s)L} \mathbb{P}L e^{s\mathbb{Q}L} \mathbb{Q}L x_j \, ds
	\label{eq:mori-zwanzig}
\end{equation}

A equação acima expressa a \textit{equação de Mori-Zwanzig}.

\subsection{Análise termo a termo} \label{sec:ch03_analise_termo_a_termo}
O primeiro termo é dado por:  
\begin{equation}
	e^{tL} \mathbb{P}L x_j
	\label{eq:ch03-primeiro-termo-mz}
\end{equation}

Observe que:  
\begin{equation*}
	Lx_j = \sum_i R_i\left(\frac{\partial}{\partial x_i}\right)x_j = R_j(x)
\end{equation*}

Portanto,  
\begin{equation*}
	\mathbb{P}Lx_j = \mathbb{E}[R_j(x)\,|\,\hat{x}] \quad \text{Note que esta é uma função exclusivamente de $\hat{x}$.}
\end{equation*}

Com isso, podemos concluir que:  
\begin{equation*}
	e^{tL}\mathbb{P}Lx_j = \bar{R}_j\left(\hat{\varphi}(x,t)\right)
\end{equation*}

Mais do que isso: o primeiro termo representa a dinâmica própria do sistema nas variáveis resolvidas. Além disso, trata-se de um termo markoviano, pois depende apenas do estado atual do sistema no tempo $t$.

Para o segundo termo, definimos:
\begin{equation*}
	w_j = e^{t\mathbb{Q}L} \mathbb{Q}L x_j
\end{equation*}

Por definição, temos:
\begin{align}
	\frac{\partial}{\partial t} w_j(x,t) & = \mathbb{Q}L w_j(x,t),\\
	w_j(x,0)                             & = \mathbb{Q}L x_j = (I - \mathbb{P})R_j(x) = R_j(x) - \mathbb{E}[R_j \mid \hat{x}]. 
    \label{eq:ch03-mori-zwanzig-dinamica-ortogonal}
\end{align}

Note que $w_j(x,0) = \mathbb{Q}Lx_j = R_j(x) - \mathbb{E}[R_j(x) \mid \hat{x}]$ representa a \textit{parte flutuante} da variável $R_j(x)$, ou seja, o componente imprevisível dado $\hat{x}$. Essa parte evolui de acordo com as \textit{dinâmicas ortogonais}, de modo que $\mathbb{P} w_j(x,t) = 0$ para todo $t$, mantendo o termo como um ruído puramente não resolvido ao longo do tempo.

Mais especificamente, o subespaço do ruído (\textit{noise subspace}) é formado pelas componentes das funções que são ortogonais às funções de $\hat{x}$, geralmente, isso corresponde a termos que dependem de $\tilde{x}$. 

O terceiro termo, dado por:
\begin{equation*}
	\int_0^t e^{(t-s)L} \mathbb{P}L e^{s\mathbb{Q}L} \mathbb{Q}L x_j
\end{equation*}
é classificado como o termo de memória  (\textit{memory term}), já que este envolve a integração de quantidades que dependem de estados anteriores ao atual.

Tomemos que $\mathbb{P}$ seja projete na extensão dos polinômios hermitianos $H-1, H_2, \ldots$ com argumentos em $\hat{x}$. Assim, para dada função $\psi$, temos que: $\mathbb{P}\psi = \sum (\psi, H_k)H_k$, assim, temos:
\begin{align*}
	\mathbb{P}Le^{s\mathbb{Q}L} \mathbb{Q}Lx_j & = \mathbb{P}L(\mathbb{P} + \mathbb{Q})e^{s\mathbb{Q}L} \mathbb{Q}Lx_j                           \\
	                                           & = \mathbb{P}L\mathbb{Q}e^{s\mathbb{Q}L} \mathbb{Q}Lx_j                                          \\
	                                           & = \sum_k \langle L\mathbb{Q}e^{s\mathbb{Q}L} \mathbb{Q}Lx_j, H_k(\hat{x}) \rangle H_k(\hat{x}). 
\end{align*}

O produto interno é definido como um valor esperado com respeito à densidade de probabilidade inicial. Vamos assumir que $L$ é antissimétrico, ou seja, $(u, Lv) = -(Lu, v)$, então:
\begin{align*}
	(L Q e^{s Q L} Q L x_j, H_k(\hat{x})) 
	  & = - (Q e^{s Q L} Q L x_j, L H_k)  \\
	  & = - (e^{s Q L} Q L x_j, Q L H_k). 
\end{align*}

Tanto $Q L x_j$ quanto $Q L H_k$ estão no subespaço de ruído, e $\mathbb{E}^{s Q L} Q L x_j$ é uma solução no tempo $s$ da equação de dinâmica ortogonal com dados no subespaço de ruído; $P L e^{s Q L} Q L x_j$ é então uma \textbf{soma de covariâncias temporais de ruídos}.

